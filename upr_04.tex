%% LyX 2.3.5.2 created this file.  For more info, see http://www.lyx.org/.
%% Do not edit unless you really know what you are doing.
\documentclass[bulgarian]{article}
\usepackage[T1,T2A]{fontenc}
\usepackage[utf8]{inputenc}
\usepackage{geometry}
\geometry{verbose,tmargin=1in,bmargin=1in,lmargin=1in,rmargin=1in}
\usepackage{textcomp}
\usepackage{amsmath}
\usepackage{amssymb}
\PassOptionsToPackage{normalem}{ulem}
\usepackage{ulem}

\makeatletter

%%%%%%%%%%%%%%%%%%%%%%%%%%%%%% LyX specific LaTeX commands.
\DeclareRobustCommand{\cyrtext}{%
  \fontencoding{T2A}\selectfont\def\encodingdefault{T2A}}
\DeclareRobustCommand{\textcyr}[1]{\leavevmode{\cyrtext #1}}


\AtBeginDocument{
  \def\labelitemiii{\(\circ\)}
}

\makeatother

\usepackage{babel}
\begin{document}
\title{Линейна алгебра, Информатика, Група 6\\
Онлайн упражнение №1}
\author{Александър Гудев}

\maketitle
\tableofcontents{}

\section{Линейни пространства}

${\displaystyle V}$ е \textbf{линейно пространство }над \textit{полето
}${\displaystyle \mathbb{F}}$, ако са изпълнени:
\begin{enumerate}
\item Групови свойства: (${\displaystyle V}$ е комутативна (=абелева) група) 
\begin{enumerate}
\item Затвореност относно събиране: ${\displaystyle \forall v,w\in V:\mathbf{v+w\in V}}$ 
\item Асоциативност на събирането: ${\displaystyle \forall u,v,w\in V:(u+v)+w=u+(v+w)}$ 
\item Съществуване на неутрален елемент: ${\displaystyle \exists\ 0\in V:\forall u\in V:0+u=u+0=u}$ 

Заб.: използваме нулата единствено като буква -- в общия случай тя
не обозначава числовата нула, позната от училище. С 0 отбелязваме
също и нулевия вектор $(0,\dots,0)$, например, като от контекста
се разбира кой символ \quotedblbase 0`` какво бележи.
\item Обратимост на всеки елемент: ${\displaystyle \forall u\in V:\exists\ (-u)\in V:u+(-u)=0}$ 
\item Комутативност на събирането: ${\displaystyle \forall u,v\in V:u+v=v+u}$
(група с това свойство се нарича \emph{абелева}) 
\begin{itemize}
\item Пример за некомутативна операция - композиция на непрекъснати функции
(внимание - те не образуват група!):
\[
{\displaystyle sin(2x)\neq2sin(x)}
\]
\end{itemize}
\end{enumerate}
\item Свойства, свързващи групата и скаларното умножение 
\begin{enumerate}
\item Затвореност относно умножение със скалар: ${\displaystyle \forall v\in V,\lambda\in F:\mathbf{\lambda v\in V}}$ 
\item Асоциативност: ${\displaystyle \forall\lambda,\mu\in F,v\in V:\lambda(\mu\ v)=(\lambda\mu)v}$ 
\item Неутралност на умножението с единица: ${\displaystyle \forall v\in V:1_{F}\ v=v}$ 
\item Дистрибутивни закони (разкриване на скоби). Нека ${\displaystyle \lambda,\mu\in F,u,v\in V}$,
тогава: 
\begin{enumerate}
\item ${\displaystyle (\lambda+\mu)v=\lambda v+\mu v}$ 
\item ${\displaystyle \lambda(u+v)=\lambda u+\lambda v}$ 
\end{enumerate}
\end{enumerate}
\end{enumerate}

\section{Примери}

Проверете, че дефиницията за ЛП е изпълнена за примерите по-долу:
\begin{itemize}
\item вектори в равнината, пространството, в 4D, ... -- наредени n-орки:
\[
\left\{ (a_{1},\dotsc,a_{n})\in\mathbb{F}^{n}\right\} 
\]
\item полиноми (с числови (засега) коефициенти): 
\[
\left\{ a_{0}+a_{1}x+\dotsc+a_{n}x^{n}\ |\ a_{i}\in\mathbb{F},n\in\mathbb{N}\right\} 
\]

\begin{itemize}
\item Също, полиноми от степен ${\displaystyle \leqslant n}$ за някое фиксирано
${\displaystyle n}$: $\mathbb{F}^{\le n}[x]$. A полиномите от степен
\emph{точно} $n$?
\begin{itemize}
\item $\mathbb{F}^{\le3}[x]=\{ax^{3}+bx^{2}+cx+d|a,b,c,d\in\mathbb{F}\}$\\
$(x^{3}+2x)+(-x^{3}-7)=2x-7$
\end{itemize}
\end{itemize}
\item всички реални функции в даден интервал: ${\displaystyle \left\{ f:[a,b]\rightarrow\mathbb{R}\right\} }$
\begin{itemize}
\item $\underbrace{(f+g)}_{\text{събиране на \textbf{функции}}}(x)=\underbrace{f(x)+g(x)}_{\text{събиране на \textbf{числа}}}$
\item $\underbrace{(\lambda\cdot f)}_{\text{умножение на функция със скалар}}(x)=\text{\ensuremath{\underbrace{\lambda\cdot f(x)}_{\text{умножение на \textbf{числа}}}}}$
\item $(-f)(x)=-\underbrace{f(x)}_{\text{\ensuremath{\mathbb{R}}}}$
\end{itemize}
\item всички непрекъснати функции в даден интервал
\item всички безкрайни числови редици: $\{(a_{1},a_{2},a_{3},\,\dots\,,a_{n},\,\dots\,)\,|\,a_{i}\in\mathbb{R}\forall i\}$
\end{itemize}

\section{Подпространства}
\begin{itemize}
\item Нека $V$ е линейно пространство. Ще казваме, че $W$ е подпространство
на $V$, ако:
\begin{enumerate}
\item $W\subseteq V$
\item $W$ е затворено относно събиране на вектори 
\item $W$ е затворено относно умножение със скалар.
\end{enumerate}
\item Казахме, че $\mathbb{F}^{\le2}[x]$ и $\mathbb{F}^{\le5}[x]$ са пространства.
Вярно ли е, че ${\mathbb{F}^{\le2}[x]\le\mathbb{F}^{\le3}[x]}$?
\begin{itemize}
\item Зависи как дефинираме строго $\mathbb{F}^{\le n}[x]$. Ако разглеждаме
полиномите като наредени $n$-орки, вж. примера по-долу с $\mathbb{R}^{2}\text{ и \ensuremath{\mathbb{R}^{3}}}.$Ако
ги дефинираме като функции над множеството $\mathbb{F}$, тогава да.
\end{itemize}
\end{itemize}
\begin{quotation}
{\Large{}Декартово произведение} (напомняне) 

Ако A и B са множества, то под ${\displaystyle A\times B}$ разбираме
множеството от всички наредени двойки:${\displaystyle \{\,(a,b)\,|\,a\in A,b\in B\}}$ 
\end{quotation}
\begin{itemize}
\item Казахме също, че $\mathbb{R}^{3}$- тримерното пространство, и $\mathbb{R}^{2}$-
равнината -- са линейни пространства.
\begin{itemize}
\item Вярно ли е, че $\mathbb{R}^{2}\le\mathbb{R}^{3}$?
\begin{itemize}
\item $R^{2}=\{(a,b)\;|\;a,b\in R\}$
\item $R^{3}=\{(a,b,c)\;|\;a,b,c\in R\}$
\item $A\subseteq B\Leftrightarrow\forall a\in A:a\in B$
\item Внимание!$R^{2}\not\subseteq R^{3}$, следователно и $\mathbb{R}^{2}\not\le\mathbb{R}^{3}$
\begin{itemize}
\item Наредените двойки \textbf{не са }наредени тройки. 
\end{itemize}
\end{itemize}
\item Вярно ли е, че $W=\{(x,0,z)\,|\,x,y\in\mathbb{R}\}$ е подпространство
на $\mathbb{R}^{3}$?
\begin{itemize}
\item $W\subseteq\mathbb{R}^{3}$
\item $(x,0,z)+(a,0,b)=(x+a,0+0,z+b)\boldsymbol{\in W},\;\boldsymbol{\forall}x,z,a,b\in\mathbb{R}$
\item $\lambda(x,0,z)=(\lambda x,0,\lambda z)\boldsymbol{\in W},\;\boldsymbol{\forall}x,z,\lambda\in\mathbb{R}$
\end{itemize}
\end{itemize}
\item $V=\{f:\text{\ensuremath{\mathbb{R\to\mathbb{R}}}}\,|\,f\text{ е непрекъсната}\}$
\begin{itemize}
\item $W=\{f:\mathbb{R\to R}\,|\,f(53)=0\text{ и }f\text{ е непрекъсната}\}$
подпространство ли е на $V$?
\begin{itemize}
\item $W\subseteq V$- изпълнено!
\item $f,g\in W,f+g\in W\Leftrightarrow(f+g)(53)=0\Leftrightarrow f(53)+g(53)=0\Leftrightarrow0+0=0$
\item $\lambda\in\mathbb{R},f\in W,\lambda f\in W\Leftrightarrow(\lambda f)(53)=0\Leftrightarrow\lambda f(53)=0\Leftrightarrow\lambda0=0\Leftrightarrow0=0$
\end{itemize}
\item Ами $W'=\{f:\mathbb{R\to R}\,|\,f(53)=2\text{ и f е непрекъсната}\}$?
\begin{itemize}
\item $W'\subseteq V$- изпълнено!
\item $f,g\in W',f+g\in W'\Leftrightarrow(f+g)(53)=2\Leftrightarrow f(53)+g(53)=2\Leftrightarrow2+2=2$
- не е изпълнено!
\end{itemize}
\end{itemize}
\end{itemize}

\section{Линейна обвивка}

Знаем какво са подпространства - а как можем да си ги създаваме сами?
\begin{itemize}
\item Разглеждаме ${\displaystyle V_{2}=\mathbb{R}^{2}}\text{ и }V_{3}=\mathbb{R}^{3}$
(над полето ${\displaystyle \mathbb{R}}$): 
\begin{itemize}
\item Ако $W\le V_{2}$ е подпространство и $(1,2)\in W$, какво най-малко
още съдържа $W$? \textbf{Затвореност!}
\begin{itemize}
\item $\{\lambda(1,2)\,|\,\lambda\in\mathbb{R}\}\subseteq W$
\end{itemize}
\item Ако $W\le V_{3}$ е подпространство и $(0,1,0),(2,0,-1)\in W$, какво
най-малко още съдържа $W$?
\begin{itemize}
\item $(0,1,0)+(2,0,-1)=(2,1,-1)\in W$
\item $W$със сигурност съдържа равнината, определена от точките $(0,0,0),(0,1,0),(2,0,-1)$
\end{itemize}
\end{itemize}
\item Сега нека $V=\mathbb{F}[x]$ е пространството от полиномите с коефициенти
от $\mathbb{F}$ (да речем, $\mathbb{R}\text{ или }\mathbb{Q}$).
\begin{itemize}
\item Ако $W$е подпространство, съдържащо $3x^{2}$ и $-7x^{7}$, какво
най-малко още съдържа $W$?
\begin{itemize}
\item $\lambda3x^{2},\mu(-7x^{7})\in W$, за всички $\lambda\text{ и }\mu\in\mathbb{F}$
\item $\lambda x^{2}+\mu x^{7}\in W$, за всички $\lambda\text{ и }\mu\in\mathbb{F}$
\end{itemize}
\end{itemize}
\item Ако $V$е пространство над полето$\mathbb{F}$ и \textbf{$\{a_{i}\}_{i=1}^{n}\text{\ensuremath{\subseteq V}}$
}са вектори от него, как да дефинираме \quotedblbase най-малкото``
подпространство, което ги съдържа?
\begin{itemize}
\item $l(\{a_{i}\}):=\{\sum_{i=1}^{n}\lambda_{i}a_{i}\,|\,\lambda_{i}\in\mathbb{F}\;\forall i\in\{1,\dots,n\}\}$
\end{itemize}
\item Какво е $l(a_{1},a_{2},a_{3})$, ако $a_{1}=(0,1,0);\;a_{2}=(0,-\sqrt{10},0),a_{3}=(7,0,0)$?

\begin{align*}
l(a_{1},a_{2},a_{3}) & =\{\lambda_{1}(0,1,0)+\lambda_{2}(0,-\sqrt{10},0)+\lambda_{3}(7,0,0)\,|\,\lambda_{i}\in\mathbb{R}\}\\
 & =\{(0,a,0)+(0,b(-\sqrt{10}),0)+(7c,0,0)|a,b,c\in\mathbb{R}\}\\
 & =\{(a,b,0)\;|\;a,b\in\mathbb{R}\}
\end{align*}

\begin{itemize}
\item Равнината $Oxy$ в тримерното пространство!
\item Вярно ли е, че $l(a_{1},a_{2},a_{3})=l(a_{1},a_{3})$? С най-малко
колко вектора можем да \quotedblbase опишем`` това подпространство?
\end{itemize}
\end{itemize}

\section{Линейна независимост на вектори }
\begin{itemize}
\item Казваме, че векторите ${\displaystyle \{a_{i}\}_{i=1}^{n}}$ са\textbf{
линейнонезависими} (ЛНЗ), ако за всеки избор на коефициенти ${\displaystyle \{\lambda_{i}\}_{i=1}^{n}}$
е изпълнено: 
\[
\sum\limits _{i=1}^{n}\lambda_{i}a_{i}=0\rightarrow\forall i:\lambda_{i}=0
\]
Тоест, ако единственият начин да получим ${\displaystyle 0}$, събирайки
тези вектори, умножени с число, е да умножим всичките с 0.\\
Или, ако не можем да изразим никой от векторите като линейна комбинация
на останалите.
\begin{itemize}
\item ако можехме, щеше поне един от коефициентите в линейната комбинация
да е ненулев, за да го прехвърлим от другата страна на равенството.
\end{itemize}
\end{itemize}
\begin{enumerate}
\item Докажете, че следните вектори от $\mathbb{F}^{4}$ са ЛНЗ:
\begin{itemize}
\item $a=(-1,2,3,-2);b=(2,1,-4,-3);c=(1,3,-2,-3)$

Взимаме дефиницията за линейна независимост: $\sum\limits _{i=1}^{n}\lambda_{i}a_{i}=0\rightarrow\forall i:\lambda_{i}=0$,
и я прилагаме директно върху условието:
\[
\lambda_{1}a+\lambda_{2}b+\lambda_{3}c=0\Longleftrightarrow
\]
\[
\lambda_{1}(-1,2,3,-2)+\lambda_{2}(2,1,-4,-3)+\lambda_{3}(1,3,-2,-3)=0\Longleftrightarrow
\]
\[
-1\lambda_{1},2\lambda_{1},3\lambda_{1},-2\lambda_{1})+(2\lambda_{2},1\lambda_{2},-4\lambda_{2},-3\lambda_{2})+(1\lambda_{3},3\lambda_{3},-2\lambda_{3},-3\lambda_{3}=0\Longleftrightarrow
\]
\[
-1\lambda_{1}+2\lambda_{2}+1\lambda_{3},2\lambda_{2}+1\lambda_{1}+3\lambda_{3},-2\lambda_{3},-4\lambda_{2}+3\lambda_{1},-2\lambda_{1}-3\lambda_{2}-3\lambda_{3}=0\Longleftrightarrow
\]
\[
-1\lambda_{1}+2\lambda_{2}+1\lambda_{3}=0;2\lambda_{2}+1\lambda_{1}+3\lambda_{3}=0;-2\lambda_{3},-4\lambda_{2}+3\lambda_{1}=0;-2\lambda_{1}-3\lambda_{2}-3\lambda_{3}=0
\]

Получената (хомогенна) система решаваме, както си знаем:
\[
\left(\begin{array}{ccc}
-1 & 2 & 1\\
2 & 1 & 3\\
-2 & -4 & 3\\
-2 & -3 & -3
\end{array}\right)\to\left(\begin{array}{ccc}
-1 & 2 & 1\\
0 & 5 & 5\\
0 & -6 & 1\\
0 & 7 & 5
\end{array}\right)\to\left(\begin{array}{ccc}
-1 & 2 & 1\\
0 & 1 & 1\\
0 & 1 & 0\\
0 & 1 & 0
\end{array}\right)\to\left(\begin{array}{ccc}
1 & 0 & 0\\
0 & 0 & 1\\
0 & 1 & 0
\end{array}\right)
\]
И резултатът е $1\lambda_{1}=0,\lambda_{3}=0,\lambda_{2}=0\Longleftrightarrow\lambda_{i}=0\;\forall i\in\{1,2,3\}$,
тоест $a,b,c$ са ЛНЗ.

\bigskip{}
\emph{Следващите два примера проверете }\emph{\uline{сами}}\emph{!}
\item $a=(1,-1,2,3);b=(2,-2,1,1)$
\item $a=(1,2,-1,3);b=(-1,-2,1,1)$
\end{itemize}
\item Проверете, че $a,b,c$ са ЛНЗ вектори от $\mathbb{F}^{3}:$
\begin{itemize}
\item $a=(1,1,1),\;b=(1,1,2),\;c=(1,2,3)$?
\begin{itemize}
\item Вярно ли е, че $\mathbb{F}^{3}=l(a,b,c)$?\\
Три вектора -- в пространството $\mathbb{F\times\mathbb{F\times F}}$!\\
Такива вектори, които са едновременно ЛНЗ и тяхната обвивка е цялото
разглеждано пространство, наричаме \textbf{\emph{базис}}\emph{ }на
пространството.
\item Училищен базис в 3D: $(1,0,0);\;(0,1,0);\;(0,0,1)$
\item Какво означава $\mathbb{F}^{3}=l(a,b,c)$? Че всеки вектор $v\in\mathbb{F}$може
да се представи като комбинация на $a,b,c$!
\end{itemize}
\end{itemize}
\item Докажете, че $a,b,c$ образуват базис на $\mathbb{F}^{3}$, и намерете
координатите на вектора $v=(2,2,2)$ в този базис:
\begin{itemize}
\item $a=(1,1,1),\;b=(1,1,2),\;c=(1,2,3)$
\item $a=(2,2,-1);\;b=(2,-1,2);\;c=(-1,2,2)$
\item $a=(1,2,3);\;b=(2,5,7);\;c=(3,7,11)$
\end{itemize}
\end{enumerate}

\end{document}
